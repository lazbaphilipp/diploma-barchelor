{\Large\normalfont\bfseries \hfil Аннотация}
\vspace{1em}
\thispagestyle{empty}

Дипломная работа на тему: <<Система управления и телеметрии на основе технологий IoT>>

Работа состоит из введения, семи глав, заключения, списка источников и приложения.

Введение раскрывает актуальность темы данной ВКР и описываются задачи данной работы.

В первой главе формируются технические задания на систему и её составляющие.

Во второй главе проводится разработка подсистемы связи базовой станции.

В третьей проводится разработка мобильного оконечного устройства \linebreak(МОУ).

Четвертая глава посвящена экспорту документации на проект.

В пятой главе описан процесс разработки компактной антенны для МОУ.

В шестой главе изложены методики измерения параметров приёмного и передающего трактов разрабатываемых устройств.

Седьмая глава посвящена оценке дальности связи в разработанной системе.

В заключении подводятся итоги о проделанной работе, делаются выводы.

В Приложении приведён листинг кода, использованного для расчётов в главе 7.

Работа содержит \pageref*{LastPage} страниц, \totalfigures\ таблицы, \totaltables\ рисунка и одно приложение. Для написания данной работы было использовано 14 источников. 

\newpage
{\Large\normalfont\bfseries \hfil Annotation}
\vspace{1em}
\thispagestyle{empty}

Degree work on the topic: <<Control and telemetry system based on IoT \linebreak technologies>>.

The work consists of an introduction, six chapters, a conclusion, a list of sources and appendix.

The introduction reveals the relevance of the chosen topic of work and describes the tasks of this work.

In the first chapter, technical specifications for the system and its components are formed.

In the second chapter, the development of the communication subsystem of the base station is carried out.

The fourth chapter is devoted to the export of project documentation.

The fifth chapter describes the process of developing a compact antenna for the MTD.

The sixth chapter describes the methods of measuring the parameters of the receiving and transmitting paths of the developed devices.

The seventh chapter is devoted to the evaluation of the communication range in the developed system.

In conclusion, the results of the work done are summarized, conclusions are drawn.

The Appendix provides a listing of the code used for calculations in Chapter 7.

The work consists of \pageref*{LastPage} pages, \totaltables\ tables, \totalfigures\  figures, 1 appendix. In this work 14 sources were used.
