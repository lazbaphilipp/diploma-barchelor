\ssect{Введение} \label{chap:introduction}
На сегодняшний день широкое распространение в различных отраслях народного хозяйства получили технологии интернета вещей (англ. Internet of Things – IoT). Системы дистанционного управления и телеметрии позволяют создавать цифровых двойников промышленных объектов, упрощают проведение масштабного экономического и эксплуатационного планирования. 

В России поддержка инициатив по развитию данной отрасли осуществляется в рамках Национальной Технологической Инициативы (НТИ) -- НКО, созданной Постановлением председателя Правительства РФ Дмитрия Медведева для объединения представителей бизнеса и экспертных сообществ для развития в России перспективных технологических рынков и отраслей, которые могут стать основой мировой экономики. Целью этого проекта является поддержка российских участников высокотехнологичных рынков технологий, борьба за лидерство на которых состоится на горизонте ближайших 20 лет в процессе цифровизации мировой экономики.\cite{nti2035}

Одним из четырнадцати центров компетенций по сквозным технологиям, созданных в интересах НТИ и программы “Цифровая экономика РФ” является центр НТИ “Сенсорика” на базе НИУ МИЭТ. В “Сенсорике” разрабатывают оптоэлектронные устройства для космических аппаратов, датчиков для сетей IoT, биосенсорами и системами компьютерного зрения, которые будут использоваться в производстве беспилотных автомобилей и летательных аппаратов. 

Целью данной работы является создание ряда устройств, способных закрывать ряд запросов рынков  Хэлснет, Фуднет, Энерджинет, Технет и Хоумнет.

Наиболее перспективными на текущий момент являются технологии, использующие существующую инфраструктуру мобильных сетей (3G, 4G, 5G) и использующие технологию LoRa (Long Range). 

Первые выгодно применять в приложениях, требовательных к ширине канала связи,но не предъявляющих жестких требований к автономности и дальности связи. Это, например, технологии городского видеонаблюдения в реальном времени или носимые системы кардиомониторинга. 

Вторые же больше подходят для систем с более низкой скоростью обмена данными, но более требовательных к энергоэффективности устройств и дальности связи. Это, например, системы сбора данных для ЖКХ, сельского хозяйства или наблюдения за состоянием газопроводов в удаленных областях, где нет возможности подвести линии электропередач. 

В рамках данной работы будут разработаны подсистема связи для базовой станции и оконечное устройство для сетей второго типа.

